\documentclass[12pt]{article}

% XeLaTeX
\usepackage{fontspec}
\usepackage[german]{babel}
% images
\usepackage{graphicx}
% spacing
\usepackage[onehalfspacing]{setspace}
% layout
\usepackage{placeins}
\setlength{\parindent}{0em}
\usepackage[a4paper,left=4cm,right=3cm,top=2.5cm,bottom=2.5cm]{geometry}
\usepackage{soul}
\usepackage{changepage}
\setcounter{secnumdepth}{5}
\setcounter{tocdepth}{3}


% sphinx documentation
% \usepackage{standalone}
% \usepackage{sphinx}
\usepackage{pdfpages}

% BibLaTeX
\usepackage[bibencoding=auto, backend=biber, style=apa]{biblatex}
\bibliography{refs.bib}
\DeclareLanguageMapping{german}{german-apa}
\usepackage{csquotes}

\begin{document}
%%%%%%%%%%%%%%%%%%%%%%%%%
%%% TITELSEITE %%%%%%%%%%
%%%%%%%%%%%%%%%%%%%%%%%%%
\begin{titlepage}
\begin{singlespacing}
\newcommand{\HRule}{\rule{\linewidth}{0.5mm}} % Defines a new command for the horizontal lines, change thickness here

\begin{center}
\textsc{\LARGE Universität Potsdam}\\[0.5cm]
\textsc{\large Department Linguistik}\\[1.5cm]
\HRule \\[0.4cm]
{\huge \bfseries Dokumentation des Skills\\[0.2cm] “Mensa-Auskunft”}\\[0.5cm]
{\Large Projektbericht über eine Funktionserweiterung von Amazons digitalem Assistenten \emph{Alexa}}\\[0.2cm]
\HRule \\[1cm]

\end{center}
\begin{tabular}{rl}
  \emph{Modul:}&
  ANW A / AM 4 – Anwendungen der Computerlinguistik\\[0.1cm]
  \emph{Seminar:}&
  Praktische Dialogmodellierung: Ein Dialogsystem erstellen\\[0.1cm]
  \emph{Semester:}&
  Sommersemester 2019\\[0.1cm]
  \emph{Dozent:}&
  Prof.~Dr.~David \textsc{Schlangen}\\[2cm]

  \emph{Autoren:}&
  Bogdan \textsc{Kostić},
  MATRIKELNUMMER\\&
  
  Maria \textsc{Lomaeva},
  MATRIKELNUMMER\\&
  
  Monique \textsc{Noss},
  MATRIKELNUMMER\\&
  
  Olha \textsc{Zolotarenko},
  MATRIKELNUMMER\\[1.75cm]
\end{tabular}

\begin{center}
{\large \today}\\[2cm]

\includegraphics[width=3cm]{uni_potsdam_logo.pdf}\\[1cm]
\end{center}

\end{singlespacing}
\end{titlepage}

%%%%%%%%%%%%%%%%%%%%%%%%%
%%% INHALTSVERZEICHNIS %%
%%%%%%%%%%%%%%%%%%%%%%%%%
\tableofcontents
\thispagestyle{empty}
\newpage  

%%%%%%%%%%%%%%%%%%%%%%%%%
%%% INHALT %%%%%%%%%%%%%%
%%%%%%%%%%%%%%%%%%%%%%%%%
\section{Einleitung}
\emph{Alexa}, der sprachgesteuerte, digitale Assistent von Amazon, kann seinem Nutzer mit einer Bandbreite an Standardaufgaben wie der Wettervorhersage oder dem Stellen eines Timers helfen.
\emph{Alexa} ist jedoch dazu fähig, weit mehr als nur Standardaufgaben zu lösen.
Amazon bietet Softwareentwicklern und Drittanbietern nämlich die Möglichkeit, die Funktionen des digitalen Assistenten mit sogenannten Skills zu erweitern.
Die an uns gestellte Aufgabe war es also, einen eigenen Skill für \emph{Alexa} zu entwickeln.

Unser Skill stellt dem Nutzer Informationen über die Hochschulmensen Deutschlands bereit.
Der Nutzer kann mithilfe des Skills dementsprechend nach dem Speiseplan einer Mensa, der Adresse einer Mensa, einer Auflistung der Mensen in einer bestimmten Stadt sowie nach der zu ihm nächst gelegenen Mensa fragen.
Um diese Fragen beantworten zu können, konsultiert unser Skill die OpenMensaAPI, aus der die benötigten Informationen extrahiert werden.


\section{Projektziele und Funktionen}
Im Nachfolgenden werden die erreichten Projektziele und Anforderungen erläutert. 
Darüber hinaus werden die Funktionen des Skills anhand von Beispieldialogen gezeigt.

\subsection{Anwendungsbereich und Zielgruppe}
Bei einem Mensa-Skill, dessen Hauptaufgabe es ist, über den Essensplan einer Mensa zu informieren, sind vor allem Studierende, aber auch Dozierende die Zielgruppe.
Eine kleine, aber keinesfalls unwichtige Teilgruppe dieser doch recht großen Zielgruppe sind blinde Studierende und Beschäftigte der Universitäten.
Dieser Skill ermöglicht es Menschen mit Sehbehinderung leichter an Informationen über das Essensangebot an deutschen Hochschulen zu kommen und macht damit die deutsche Hochschullandschaft noch ein wenig barrierefreier.

Eine weitere kleine Zielgruppe dieses Skills könnten Beschäftigte eines Studienakkreditierungsinstituts sein, die oft durch ganz Deutschland reisen und viel Zeit an den unterschiedlichsten Hochschulen verbringen.
Dementsprechend könnten diese Beschäftigte an den Speiseplänen vieler verschiedener Universitäten und Fachhochschulen interessiert sein.
Da wir versucht haben, die ganze Bandbreite der OpenMensa API zu nutzen und somit Informationen für 490 verschiedene Mensen bereitstellen können, ist dieser Skill dazu in der Lage, diesem Interesse entgegenzukommen.

Das Informationsbedürfnis dieser Zielgruppen ist es einerseits, eine grobe Übersicht über das Angebot zu erhalten, andererseits aber auch nach bestimmten Zutaten gefilterte Ergebnisse zu erhalten, falls Allergien oder besondere Ernährungsweisen vorliegen (z.B. Veganismus).
Auch der Preis eines bestimmten Gerichtes könnte für eine finale Auswahl entscheidend sein, da das Budget eines Studierenden selten großzügig ausfällt.
Diese und weitere verschiedene Funktionen wurden eingebaut, um die in der OpenMensa API angebotenen Daten ausgiebig zu nutzen.

Dazu gehören auch die folgenden Lokationsfunktionen:
\begin{itemize}
  \setlength\itemsep{0em}
  \item Adresse von Mensen
  \item Koordinaten, um die nächste Mensa zu finden
  \item Mensen nach Stadt finden
\end{itemize}

Neben den oben genannten Funktionen, die durch selbsterstellte Intents des Skills umgesetzt wurden, bietet Amazon den Skill-Entwicklern an, dynamische, auf den Benutzer zugeschnittene Funktionen einzubauen. 
Ein Beispiel hierfür sind die "Persist Attributes" (= "bleibende Attribute").
Persist Attributes dienen dazu, bestimmte Daten in einer Session zu speichern, damit sie in der nächsten wieder abgerufen werden können.
So müsste ein Benutzer nicht bei jedem Launch mit erwähnen, für welche Mensa ein Essensplan ausgegeben werden soll.
Während die Implementierung und Einbettung über AWS DynamoDB einfach vorgenommen werden kann, könnten diese beim Entwickler Kosten verursachen, weshalb wir in diesem Skill auf die Benutzung dieser Attribute verzichtet haben.
Auch könnten User Accounts angelegt werden, um persönliche Vorlieben des Users speichern zu können. 

Eine weitere mögliche Verbesserung haben wir während der zweiten Projektphase, also nach der ersten Vorstellung der Projekte, umsetzen können.
Es handelt sich hierbei um syntaktisches Parsing der Gerichte, um nur den Kopf der “Gerichtsphrase” auszugeben und somit die Antworten des Skills zu kürzen, um den Nutzer nicht mit Informationen zu überhäufen.
Aus demselben Grund werden nun immer nur vier Gerichte in einem Prompt vorgelesen.
Möchte der Nutzer weitere Gerichte erfahren, kann einfach "weiter" gescrollt werden. Mit einem Intent für Details können dagegen Details zu einem Gericht erfragt werden.

Für uns war es wichtig, die gesamte Breite der API zu nutzen, um dem Nutzer so viele Funktionen wie möglich zu bieten.
Nichtsdestotrotz findet die In-App-Kommunikation schnell ein Ende, wenn der Benutzer ein mögliches Ziel erreicht hat oder ein Fehler aufgetreten ist.
Der Skill kann dann in einem One-Shot erneut gelaunched werden.

\subsubsection{Essensplan}
Gibt den Essensplan einer bestimmten Mensa für ein bestimmtes Datum aus. Optional kann eine Zutat angegeben werden, die enthalten sein soll. 

Beispieläußerungen:
\begin{quote}
“lies mir den plan für \{date\} vor”\\
“gibt es \{date\} \{ingredient\} gerichte in der \{mensa\_name\}”\\
“was gibt's in der mensa”
\end{quote}

\paragraph{Gerichte ohne Zutat}~\\
Sucht Gerichte in einer bestimmten Mensa für ein bestimmtes Datum ohne bis zu zwei Zutaten.

Beispieläußerungen:
\begin{quote}
“suche für \{date\} \{synonyms\_gericht\} ohne \{ingredient\} in \{mensa\_name\}”\\
“gibt es \{date\} \{synonyms\_gericht\} ohne \{ingredient\}”\\
“nach \{synonyms\_gericht\} ohne \{ingredient\} bitte”\\
\end{quote} 

\paragraph{Gerichte mit Zutat}~\\
Sucht Gerichte in einer bestimmten Mensa für ein bestimmtes Datum mit bis zu zwei Zutaten. 

Beispieläußerungen:
\begin{quote}
“suche für \{date\} \{synonyms\_gericht\} mit \{ingredient\} in \{mensa\_name\}”\\
“gibt es \{date\} \{synonyms\_gericht\} mit \{ingredient\}”\\
“nach \{synonyms\_gericht\} mit \{ingredient\} bitte”\\
\end{quote}

\subsubsection{Preis der Gerichte}
Der Skill gibt den Preis für ein Gericht zurück.

Beispieläußerungen:
\begin{quote}
“preis für nummer \{number\}”\\
“wie viel kostet das \{number\} für \{user\_group\}”\\
“wie teuer ist das \{number\} gericht”\\
\end{quote}

\subsubsection{Adresse der Mensa}
Die Adresse einer Mensa wird vorgelesen. 

Beispieläußerungen:
\begin{quote}
“zeige mir die adresse der \{mensa\_name\}”\\
“standort \{mensa\_name\}”\\
“adresse \{mensa\_name\}”\\
\end{quote}

\subsubsection{Mensas in einer Stadt}
Listet die Mensas in einer genannten Stadt auf.

Beispieläußerungen:
\begin{quote}
“welche mensas gibt es in \{city\}”\\
“suche mensas in \{city\}”\\
“gibt es mensas in \{city\}”\\
\end{quote}

\subsubsection{Mensa in der Nähe}
Findet die Mensa, die dem Benutzerstandort am nächsten ist.

Beispieläußerungen:
\begin{quote}
“wo ist die nächste mensa”\\
“ich bin hungrig”\\
“welche mensa ist in der nähe”\\
\end{quote}

\subsection{Szenarien und Beispieldialoge}
Im nachfolgenden Absatz werden Beispieldialoge für jeden Intent aufgelistet, die Skillerfolge und -misserfolge darstellen sollen, um die Funktionalität des Skills zu verdeutlichen.
Diese werden jeweils anhand von spezifischen Szenarien veranschaulicht und erklärt.

\subsubsection{Beispieldialoge: Essensplan und Preise erfahren}
Die am häufigsten genutzte Funktion eines Mensa-Skills wird voraussichtlich das Abfragen des Essensplans für einen bestimmten Tag in der eigenen Mensa sein. 
Hierbei kann der Nutzer Zutaten inkludieren, um die Auswahl von Anfang an einzuschränken.
Wenn eine Auswahl getroffen wurde, wird der Benutzer gefragt, ob er Details oder den Preis eines der Gerichte erfahren möchte.
Der Benutzer kann an dieser Stelle auch verneinen, woraufhin der Skill geschlossen wird.

\textbf{Szenario 1: Benutzer möchte den Essensplan für eine bestimmte Mensa, für einen bestimmten Tag erfragen.}\\
Wann? Jederzeit, wann immer der User in der Mensa essen möchte.\\~\\
\ul{Situation 1.1:}\\Morgen ist ein Werktag.\\
\begin{quote}
\begin{tabular}{lp{12cm}}
  U:& Alexa, öffne Mensaauskunft.\\
  A:& Willkommen bei der Mensaauskunft! Wenn du Hilfe bei der Bedienung brauchst, sag bitte HILFE. Was möchtest du wissen?\\
  U:& Was gibt es morgen in der Mensa?\\  
  A:& In welcher Mensa möchtest du essen?\\
  U:& Mensa Golm.\\
  A:& Es gibt 5  Gerichte  zur Auswahl: 1. Eintopf aus Kichererbsen, Linsen, Gemüse, Kartoffeln, 2. Eintopf aus Kichererbsen, Linsen, Gemüse,Kartoffeln, 3. Gulasch und 4. Hähnchenbrust „Piccata Milanese“. Möchtest du mehr Gerichte hören oder Details?\\ 
  U:& Mehr!\\ 
  A:& 5. Griechisches Kartoffelgericht. Möchtest du Details zu einem dieser Gerichte erfahren? Sag zum Beispiel: Details. oder: Wie viel kostet Gericht Nummer 2 für Studenten.\\
  U:& Wie viel kostet Gericht Nummer 2 für Studenten?\\
  A:& Das Gericht Eintopf aus Kichererbsen, Linsen, Gemüse,Kartoffeln und Rindfleisch kostet 1,6 Euro für Studenten.\\[0.2cm]
\end{tabular}
--- SKILL SCHLIEßT SICH ---\\
\end{quote}

\ul{Situation 1.2:}\\Morgen ist Wochenende oder Feiertag.
\begin{quote}
\begin{tabular}{lp{12cm}}
  U:& Alexa, frag Mensaauskunft, was gibt es morgen in der Mensa Golm?\\  
  A:& Sorry, für den ausgewählten Tag 2019-10-20 gibt es leider keinen Essensplan für mensa golm.\\[0.2cm]  
\end{tabular}
--- SKILL SCHLIEßT SICH ---\\
\end{quote}

\ul{Situation 1.3:}\\Die Mensa wurde nicht richtig verstanden (ASR Error) oder ist nicht in der Datenbank.

\begin{quote}
\begin{tabular}{lp{12cm}}
  U:& Alexa, frag Mensaauskunft, was gibt es morgen in der Mensa Golm?\\
  A:& Sorry, Essenspläne für mensa golum habe ich leider nicht im Angebot.\\[0.2cm]
\end{tabular}
--- SKILL SCHLIEßT SICH ---\\
\end{quote}

\ul{Situation 1.4:}\\Der Benutzer sucht nach Gerichten mit einer bestimmten Zutat.\\
\emph{-- Dialog 1.4.1 --}
\begin{quote}
\begin{tabular}{lp{12cm}}
  U:& Alexa, frag Mensaauskunft nach veganen Gerichten.\\
  A:& Für wann?\\
  U:& Morgen.\\
  A:& In welcher Mensa möchtest du essen?\\
  U:& Mensa Golm.\\
  A:& Es gibt 2 Vegan Gerichte zur Auswahl: 1. Vegane Knusperkroketten und 2. Glasnudel-Bowl. Möchtest du Details zu einem dieser Gerichte erfahren? Sag zum Beispiel: Details. oder: Wie viel kostet Gericht Nummer 2 für Studenten.\\
  U:& Stopp.\\
  A:& Guten Hunger!\\[0.2cm]
\end{tabular}
--- SKILL SCHLIEßT SICH ---\\
\end{quote}

\emph{-- Dialog 1.4.2 --}
\begin{quote}
\begin{tabular}{lp{12cm}}
  U:& Alexa, frag Mensaauskunft nach Gerichten mit Sellerie.\\
  A:& Für wann?\\
  U:& Morgen.\\
  A:& Leider gibt es keine passenden  Gerichte mit sellerie. Kann ich sonst noch helfen?\\
  U:& Nein.\\
  A:& Okay, tschüss!\\[0.2cm]
\end{tabular}
--- SKILL SCHLIEßT SICH ---\\
\end{quote}

\subsubsection{Beispieldialoge: Die nächste Mensa finden}
Ein wenig komplexer ist die Funktionalität, eine Mensa in der Nähe des Standortes des Benutzers zu finden.
Aus dem Intent können daher daher unterschiedliche Reaktionen hervorgehen, je nach Status der Standortfreigabe für das benutzte Gerät und den Skill selbst.\\

\textbf{Szenario 2: Benutzer möchte eine Mensa in seiner Nähe finden und befindet sich im Ortsteil Golm in Potsdam.}\\
Wann? Höchstwahrscheinlich nur ein Mal, wenn der User wissen möchte, welche Mensa die nächste in der Umgebung ist.\\~\\
\ul{Situation 2.1:}\\Der User hat den Gerätestandort nicht freigegeben.
\begin{quote}
\begin{tabular}{lp{12cm}}
  U:& Alexa, frag Mensaauskunft, wo die nächste Mensa ist.\\
  A:& Um die nächste Mensa zu finden, benötige ich Deinen Standort. Bitte öffne die Alexa-App, um deinen Standort mit mir zu teilen.\\[0.2cm]
\end{tabular}
--- SKILL SCHLIEßT SICH ---\\
\end{quote}

\ul{Situation 2.2:}\\Der User hat den Gerätestandort freigegeben.
\begin{quote}
\begin{tabular}{lp{12cm}}
  U:& Alexa, frag Mensaauskunft, wo die nächste Mensa ist.\\
  A:& Die nächste Mensa ist Mensa Golm in Karl-Liebknecht-Str. 24/25, 14476 Potsdam OT Golm.\\[0.2cm]
\end{tabular}
--- SKILL SCHLIEßT SICH ---\\
\end{quote}

\subsubsection{Beispieldialoge: Die Adresse einer Mensa erfahren}
Die Adresse der Mensa kann erneut erfragt werden oder es kann auch nach Adressen anderer Mensen gesucht werden.\\

\textbf{Szenario 3: Benutzer möchte die Adresse einer Mensa erfahren.}\\
\ul{Situation 3.1:}\\Die Adresse der Mensa kann gefunden werden, sie ist also in der Datenbank vorhanden.
\begin{quote}
\begin{tabular}{lp{12cm}}
  U:& Alexa, frag Mensaauskunft nach der Adresse der Mensa Golm.\\
  A:& Die Adresse der mensa golm lautet Karl-Liebknecht-Str. 24/25, 14476 Potsdam OT Golm.\\[0.2cm]
\end{tabular}
--- SKILL SCHLIEßT SICH ---\\
\end{quote}

\ul{Situation 3.2:}\\Die Adresse der Mensa kann nicht gefunden werden, sie ist also nicht in der Datenbank vorhanden. Möglich ist auch, dass ASR den User falsch verstanden hat.
\begin{quote}
\begin{tabular}{lp{12cm}}
  U:& Alexa, frag Mensaauskunft nach der Adresse der Mensa Golm.\\
  A:& Die Adresse der angefragten Mensa mensa ball um konnte leider nicht gefunden werden.\\[0.2cm] 
\end{tabular}
--- SKILL SCHLIEßT SICH ---\\
\end{quote}

\textbf{Szenario 4: Benutzer möchte erfahren, welche Mensen es in einer bestimmten Stadt gibt.}\\
\ul{Situation 4.1:}\\Der Nutzer fragt nach Mensen in einer deutschen Stadt, zum Beispiel Potsdam.
\begin{quote}
\begin{tabular}{lp{12cm}}
  U:& Alexa, frag Mensaauskunft, welche Mensas gibt es in Potsdam?\\
  A:& Es gibt die folgenden Mensas in potsdam: Mensa Am Neuen Palais, Mensa Kiepenheuerallee, Mensa Golm, Mensa Griebnitzsee, Bistro Tasty Studio Babelsberg, Ulf's Café (HPI Cafeteria), Erfrischungshalle.\\[0.2cm]
\end{tabular}
--- SKILL SCHLIEßT SICH ---\\
\end{quote}

\ul{Situation 4.2:}\\Der Nutzer fragt nach Mensen in einer Stadt, die nicht in Deutschland liegt, zum Beispiel Tokyo.
\begin{quote}
\begin{tabular}{lp{12cm}}
  U:& Alexa, frag Mensaauskunft, welche Mensas gibt es in Tokyo?\\
  A:& Leider keine Mensas in Tokyo gefunden. Du kannst eine andere Stadt in Deutschland auswählen.\\[0.2cm]
\end{tabular}
--- SKILL SCHLIEßT SICH ---\\
\end{quote}

\section{Projektorganisation}
\subsection{TeilnehmerInnen \& Aufgabenverteilung}
Das Team setzt sich aus vier Personen zusammen: Monique Noss, Bogdan Kostić, Maria Lomaeva und Olha Zolotarenko. 
Jedem Teammitglied wurde ein Aufgabenbereich zugeteilt, für den er oder sie zuständig war.
Die Aufteilung der Intents im Team war wie folgt:
\begin{itemize}
  \setlength\itemsep{0em}
  \item Bogdan: IngredientIntent, GetNearestMensaIntent, WithoutIntent, ListDishesIntent(+WithoutIntent), NextIntent
  \item Olha: AddressIntent, Chunking
  \item Maria: ListMensasIntent, Chunking
  \item Monique: ListDishesIntent, PriceIntent, DetailsIntent, NextIntent, NoIntent
\end{itemize}

Neben dem Erstellen des Codes für das Backend der Intents sollte sich auch um die jeweiligen \emph{sample utterances}, das Testen sowie das Dokumentieren des jeweiligen Intents gekümmert werden.
Die Koordination der verschiedenen Bestandteile des Skills im Code hat Monique Noss übernommen.

\subsection{Planungsdokumente \& Milestones}
Die Planungsdokumente zusammen mit den ursprünglichen Entwicklungsideen (\texttt{ideas.md}) und den dazugehörigen Milestones sind unter \texttt{mensa-skill/material/} zu finden. 

Die erste Version des Skills und der Intents wurde in der Datei \texttt{meeting1.md} dokumentiert.
Diese Version wurde in der Vorstellung der Projekte am Ende des Sommersemesters 2019 getestet und anschließend präsentiert.
Die wichtigsten Verbesserungsvorschläge, die dabei resultiert sind, waren dabei die folgenden:
\begin{itemize}
  \setlength\itemsep{0em}
  \item Aufzählungslisten verkürzen (siehe dazu auch  \texttt{chunking.md})
  \item Die Möglichkeit, nach mehr als einer Zutat im Gericht zu suchen
  \item Nach einer Mensa in der Nähe fragen zu können, statt nur alle Mensen in der Stadt aufzulisten
  \item Mehr natürliche \emph{sample utterances} einfügen
\end{itemize}

Die zweite Version des Skills Mensa-Auskunft wurde nach einer ausführlichen Besprechung der nötigen Verbesserungen implementiert, welche oben kurz zusammengefasst wurden. 
Näheres kann man der Datei \texttt{meeting2.md} entnehmen. 

Schließlich wurde der Code organisiert und kommentiert, um die Lesbarkeit zu erhöhen. 
So entstand auch eine separate Datei \texttt{lambda\_utility.py}, die alle Hilfsfunktionen für \texttt{lambda\_function.py} beinhaltet. 

\section{Entwurf des Systems, Dokumentation}
\subsection{Entwicklungsumgebung}
\begin{figure}[h]
\center
\includegraphics[width=10cm]{mermaid-diagram.png}
\caption{Benutzte Software während der Entwicklung des Skills}
\end{figure}

Abbildung 1 zeigt, mit welcher Software der Skill erstellt wurde.
Der Skill läuft unter der Pythonversion 3.6 und importiert einige Bibliotheken, die einerseits native Python-Bibliotheken sind und andererseits von dem Alexa Skills Kit zur Verfügung gestellt wurden oder installiert werden müssen.
Diese Bibliotheken müssen im selben Ordner der \texttt{lambda\_function.py} liegen, damit diese auch von AWS Lambda eingelesen werden können.
Für die Installation wurde \texttt{pip} und das Kommando \texttt{pip3 install -r requirements.txt -t .} benutzt.
Zu den Requirements gehören:
\begin{itemize}
  \setlength\itemsep{0em}
  \item ask-sdk
  \item flask
  \item flask-ask-sdk
  \item haversine
\end{itemize}

\texttt{ask-sdk} wird benötigt, um die Kommunikation mit \emph{Alexa} herzustellen.
Das ganze Skill-Building basiert auf den Bibliotheken dieser Software, dazu gehören die Intent-Klassen und Funktionen sowie das Verarbeiten des Sprach-Outputs (Prompts). 

\texttt{flask} und \texttt{flask-ask-sdk} wird zum Aufsetzen eines lokalen Servers verwendet.
Wenn der Server gestartet ist, kann das Alexa Skills Kit mit dem auf dem Computer gespeicherten Code kommunizieren, wenn dieser lokal ausgeführt wird.
Dazu wird außerdem das Tool \texttt{ngrok} benötigt, um den Port des Computers nach außen zu leiten.
Diese Tools waren essenziell für das Testen unseres Skills. 

\texttt{haversine} ist eine Bibliothek, die die Distanz zwischen zwei Punkten auf der Erde mithilfe des Breiten- und Längengrades kalkuliert.
Diese wird benötigt, um die nächste Mensa in der Nähe des Benutzerstandortes zu finden.

\subsection{Tests}
Getestet wurde der Skill mithilfe von \emph{flask} und \emph{ngrok} lokal auf unseren eigenen Rechnern, um Status- und Fehlermeldungen angezeigt bekommen zu können, sodass sie direkt nachvollzogen werden konnten. 
Hat einer der Gruppenmitglieder etwas am Code verändert, so musste dieses Gruppenmitglied einen \emph{Pull Request} auf Github eröffnen.
Damit diese Modifikation dann in den finalen Code übernommen wird, musste ein anderes Mitglied diese Änderungen sichten und testen.
Diese Praktik hat sicher gestellt, dass der Code auch bei Änderungen einwandfrei funktioniert, da diese von mindestens zwei Personen nachvollzogen und getestet werden mussten.

\subsection{Intents}
Dieser Skill verwendet insgesamt 13 Intents, davon sind sechs Intents \emph{Custom Intents} sowie sieben Intents \emph{Built-In Intents}, die von Amazon zur Verfügung gestellt werden und teilweise von uns mit eigenen \emph{sample utterances} erweitert wurden. 
Außerdem benutzt der Skill acht verschiedene Slot-Types, von denen vier zu den \emph{Amazon Built-In Slots} gehören.

\begin{figure}[ht]
\center
\includegraphics[width=\textwidth]{intent-order.png}
\caption{Intentabfolge von gelungenen Dialogen mit dem Skill}
\end{figure}

Abbildung 2 zeigt die Intentabfolge von gelungenen Dialogen mit dem Skill.
Der Intent \texttt{AMAZON.WelcomeIntent} kann mit einer One-Shot-Äußerung des Benutzers übersprungen werden.
Weicht der Nutzer von dieser Intentreihenfolge ab, so erhält er einen Hinweis, dass zunächst eine entsprechende Suche gestartet werden muss.

Im Anhang dieses Berichts befindet sich eine Dokumentation des Backends des Skills, aus der die verschiedenen Intents und Slot-Types entnommen werden können.
Diese Dokumentation wurde mithilfe des Tools \emph{Sphinx} aus den Docstrings extrahiert.

\section{Evaluation \& Projektabschluss}
\subsection{Versuchsanordnung: Probanden \& Fragebogen}
Für die Evaluation des Skills wurde ein Fragebogen bei Google Forms für die Probanden erstellt; diesen kann man unter \texttt{docs} als \texttt{Alexa Skill/Mensa-Auskunft\_Form.pdf} finden. Die Fragen wurden in Hinsicht auf folgende Punkte konzipiert:

\begin{itemize}
\setlength\itemsep{0em}
  \item Zweck der Anwendung, Zufriedenheit mit dem Produkt bzw. Entsprechnung der Erwartungen
  \item Dauer der Anwendung bis zum Erhalt der gewünschten Informationen
  \item Art von unerwünschtenn Verhalten des Skills
  \item Wahrscheinlichkeit der Weiterempfehlung
  \item Verbesserungsvorschläge
\end{itemize}

Um einschätzen zu können, welcher Nutzerkategorie die Probanden aneghören, die den Skill evaluiert haben, wurde auch nach der Erfahrung mit Sprachassistenten und Skills gefragt.
Als kritische Fragen für die allgemeine Evaluation des Skills wurden Fragen zum Informationserhalt und zur Weiterempfehlung gestellt (Fragen 2 und 6). 
Die anderen Fragen richten sich an die Verbesserung des Skills und helfen, mögliche Probleme zu entdecken und zu beheben.

\subsection{Auswertung}
Im Allgemeinen wurde der Skill von 80\% der Probanden als empfehlenswert markiert. 
Die am meisten genutzten Funktionen des Skills waren die Suche des aktuellen Essensplans und des Essensplans für einen bestimmten Tag.
Am wenigsten wurde nach der Adresse einer Mensa gesucht.
Im Durchschnitt benötigten die Probanden 3 bis 5 Minuten, um die gewünschte Information zu bekommen.
Unerwünschte Reaktionen des Skills tauchten bei 60\% der Probanden 1 bis 4 Mal, bei 20\% der Probanden 5 bis 10 Mal und bei den restlichen 20\% gar nicht auf. Die häufigsten beschriebenen unerwünschten Reaktionen waren folgende:
\begin{itemize}
\setlength\itemsep{0em}
  \item \emph{Alexa hat mich ab und zu falsch verstanden. Das liegt aber wohl eher an Alexa selbst.}
  \item \emph{Nicht verstandene Wörter}
  \item \emph{Namen der Mensen nicht verstanden besonders HU und TU Berlin, oftmals konnte ich die Infos nicht erhalten, weil etwas schief gelaufen ist, One-shots scheinen schlecht bis gar nicht zu funktionieren}
\end{itemize}

Die meisten Probanden gaben an, nur ein wenig erfahren mit \emph{Alexa} oder anderen Sprachassistenten und Skills zu sein.

Aus dem Feedback der Probanden kann entnommen werden, dass es die meisten Probleme damit gab, dass \emph{Alexa} sie nicht richtig verstanden hat.
Dieses Phänomen konnten auch wir selber beim Testen beobachten.
Es scheint als hätte Amazons \emph{Automatic Speech Recognition (ASR)} vor allem Probleme damit, Akronyme in den Äußerungen der Benutzer richtig zu erkennen.
So tauchten die Probleme größtenteils bei den Mensen auf, deren Hochschulen in der Regel mit Akronymen benannt werden, wie z.B.~\emph{TU Berlin}.
Leider hatten wir selber keinen Einfluss auf Amazons \emph{ASR}, sodass es uns nicht möglich war, dieses Problem zu beheben.

Man kann also schlussfolgern, dass der Skill allgemein schon gut funktioniert.
Es gab soweit nur positives Feedback zum Erhalt der Informationen, obwohl es aus verschiedenen Gründen bei vielen Probanden ein wenig länger dauerte.
Die Ergebnisse der Umfrage zeigen auch, dass manche Funktionen eher wenig benutzt werden.
Daher könnte man sich überlegen, diese nicht weiter auszubauen oder über die Möglichkeit der Benutzung dieser Funktionen den User mehr zu informieren.
Außerdem könnte man bei der Entwicklung neuer Funktionalitäten auch beachten, dass die meisten Probanden eher unerfahren mit Sprachassistenten waren und sich mehr Hilfe während der Unterhaltung mit dem Skill wünschen.

\subsection{Ausblick \& Verbesserungsmöglichkeiten}
TO-DO

\newpage
\addcontentsline{toc}{section}{Anhang: Dokumentation des Backends}
\includepdf[pages=5,pagecommand=\section*{Anhang: Dokumentation des Backends}]{../_build/latex/Mensa-Auskunft.pdf}
\includepdf[pages=6-12]{../_build/latex/Mensa-Auskunft.pdf}





  
  
\end{document}